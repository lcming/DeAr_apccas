\section{Performance Evaluation}
\subsection{Experiment Setup}
\label{sec:evaluation:setup}
%In this chapter, we present the performance evaluation of DeAr and demonstrate its capability of efficient arithmetic.
Two benchmark suites were prepared for the experiment.
The first one is adapted from the BLAS~\cite{blas} library, 
which contains various matrix arithmetic subprograms that are crucial in wireless communication.
The second one includes general DSP kernels (GDSPK) selected from classic DSP benchmark suites, BDTI~\cite{bdti} and DSPstone~\cite{dspstone}.
Table.~\ref{tab:op} lists the operation profiling of two benchmark suites, 
where each subprogram/kernel comprises three primitive operations, addition, multiplication and shifting.
\setlength{\textfloatsep}{0pt}
\begin{table}[h]%
\centering%
\caption{Operation profiling of two benchmark suites}%
\label{tab:op}%
\resizebox{\columnwidth}{!}%
{%
\begin{tabular}{|c|c|c|c|c|c|c|c|c|}%
\hline%
\multicolumn{9}{|c|}{\textbf{Basic linear algebra subprograms (BLAS)}} \\ \hline%
Benchmark              & AXPY   & MV     & MM      & INV      & CAXPY  & CMV  & CMM    & CINV  \\ \hline
\# add            &  32    &  56    &   48    &    75    &  128   & 132  &   90   &  88   \\ \hline
\# mul            &  32    &  64    &   64    &   172    &  128   & 144  &  108   & 114   \\ \hline
\# sht            &   0    &   0    &    0    &     0    &    0   &   0  &    0   &   0   \\ \hline
\# op             &  64    & 120    &  112    &   247    &  256   & 276  &  198   & 202   \\ \hline
\multicolumn{9}{|c|}{\textbf{General DSP kernels (GDSPK)}}                     \\ \hline
Benchmark              & FIR    & CFIR   & LPFIR   & Biquad   & IT     & DCT  & IMDCT  & FFT   \\ \hline
\# add            & 15     &  62    &   15    &    8     &  32    &  29  &   21   &  23   \\ \hline
\# mul            & 16     &  64    &    8    &    9     &   0    &  12  &   11   &  10   \\ \hline
\# sht            &  0     &   0    &    0    &    0     &  10    &   9  &    9   &   0   \\ \hline
\# op             & 31     & 126    &   23    &   17     &  42    &  50  &   41   &  33   \\ \hline
\end{tabular}%
}%
\end{table}%
Various architectures, DeAr, scalar, VLIW, and composite-ALU \cite{cascade} were evaluated for comparison. 
Note that \cite{cascade} is a class of ASIP, and we evaluated it with two ALU configurations, M-S-A and A-M-S.
\subsection{Pre-synthesis Analysis}
    Under a certain throughput requirement, operations per cycle (OPC) determines the timing constraint for hardware synthesis, 
    where higher OPC brings looser timing constraint, which results in lower cost from the post-synthesis hardware.
    Table.~\ref{tab:opc} profiles the OPC of various DSPs with the assumption that every operation consumes exactly one clock cycle.
    The OPC of the scalar processor is fixed to 1.0 because of the limitation of the single-issue datapath.
    On the contrary, benefiting from the wide issue-width and large port number of RF, VLIW gains the best OPC in all benchmarks.
    The OPC of DeAr is limited by the access pattern to the RF which either be LIFO or FIFO, 
    and thus it is not possible for DeAr to outperform the VLIW in regard to OPC.
    Nevertheless, by leveraging the HDFG-based scheduling,
    DeAr still gains approximate OPC of the VLIW, with only 1.11\% and 4.26\% loss in two benchmark suites.
    On the other hand, the OPC of composite-ALU is highly correlated with the benchmark characteristic. 
    On average, the MSA order gains better OPC than the AMS order, 
    but the former still underperforms 3-way VLIW by 12.3\% and 17.7\% in the benchmarks suites.%
\setlength{\textfloatsep}{0pt}%
\begin{table}[h]%
\centering%
\caption{Operations per cycle profiling}%
\label{tab:opc}%
\resizebox{\columnwidth}{!}%
{%
\begin{tabular}{|c|c|c|c|c|c|c|c|c|c|}%
\hline
\multicolumn{10}{|c|}{\textbf{Basic linear algebra subprograms (BLAS)}} \\ \hline
Benchmark  &  AXPY  &  MV  &  MM  &  MINV  &  CAXPY  &  CMV  &  CMM  &  CMINV  &  Average \\ \hline 
VLIW  &   1.94  &   1.85  &   1.72  &   1.44  &   1.97  &   1.89  &   1.80  &   1.76  &   1.79     \\ \hline 
DeAr  &   1.94  &   1.85  &   1.72  &   1.40  &   1.97  &   1.89  &   1.80  &   1.62  &   1.77     \\ \hline
Comp.-MSA  &   2.00  &   1.88  &   1.75  &   1.37  &   1.33  &   1.35  &   1.38  &   1.53  &   1.57     \\ \hline 
Comp.-AMS  &   1.00  &   1.00  &   1.00  &   1.02  &   1.00  &   1.00  &   1.00  &   1.04  &   1.01     \\ \hline 
Scalar  & 1.0  & 1.0  & 1.0  & 1.0  & 1.0  & 1.0  & 1.0  & 1.0  & 1.0 \\ \hline 
\multicolumn{10}{|c|}{\textbf{General DSP application kernels (GDSPK)}}                     \\ \hline
Benchmark  &  FIR  &  CFIR  &  LPFIR  &  Biquad  &  IT  &  DCT  &  IMDCT  &  FFT  &  Average \\ \hline 
        VLIW  &   1.82  &   1.91  &   1.67  &   1.55  &   1.33  &   1.61  &   1.86  &   1.38  &   1.64     \\ \hline 
        DeAr  &   1.82  &   1.91  &   1.67  &   1.55  &   1.33  &   1.47  &   1.46  &   1.32  &   1.57     \\ \hline 
        Comp.-MSA  &   1.94  &   1.34  &   1.44  &   1.42  &   1.31  &   1.14  &   1.28  &   1.06  &   1.35     \\ \hline 
        Comp.-AMS  &   1.00  &   1.00  &   1.53  &   1.21  &   1.14  &   1.61  &   1.52  &   1.27  &   1.29     \\ \hline 
        Scalar  & 1.0  & 1.0  & 1.0  & 1.0  & 1.0  & 1.0  & 1.0  & 1.0  & 1.0 \\ \hline 
    \end{tabular}%
}%
\end{table}%
\subsection{Synthesis Result and Analysis}
In this part, we present hardware synthesis results with the throughput requirement varying from 50 to 1000 MOPS, 
We implemented aforementioned designs with UMC 65nm cell library, 
and measured their area and power dissipation with Synopsys Design Compiler and Synopsys Prime Time under the timing constraint based on the average of Table.~\ref{tab:opc}.
Synthesis failures owing to timing violation reported by the tool would remain corresponding entries empty in figures.
%\subsubsection{Area}
\\\indent 
Figure~\ref{chart:area} shows the area improvement of DeAr over with other architectures.
DeAr saves 24.5\%--21.1\% of area against VLIW.
The area of the VLIW architecture is dominated by RF, 
which demands complicated interconnection among ports and registers to facilitate centralized organization.
By contrast, DeAr, which trades off the connectivity by adopting the banked RF, can thus achieve significant area improvement.
Since RF access is not the crucial issue affecting the critical path for both VLIW and DeAr, 
the correlation between throughput and save percentage is inconspicuous.
\\\indent On the other hand, DeAr achieves 21.2\%--5.6\% area improvement against composite-ALU with the MSA order.
When compared with the AMS order, the save percentage becomes 17.2\%--5.3\%.
Composite-ALU architectures reduce area by using fewer ports on the RF.
Nevertheless, the area burden caused by the centralized RF still exists.
Another drawback of composite-ALU architectures is long critical path in the ALU, 
which leads to sharp area growth as throughput increasing.
\\\indent Benefiting from the simple datapath, 
the scalar costs lowest area when throughput was low.
However, the low OPC of scalar becomes the terrible burden on achieving high throughput.
The area of scalar exceeds the one of DeAr while throughput is above 600 MOPS.
Consequently, DeAr achieves lowest area among five designs while throughput is high.%
\\\indent
The power dissipation in CMOS devices consists of dynamic power and static power.
In our experiment, the former is the dominating factor, which accounts for 98\%--70\% of the total dissipation, 
owing to the cell library characteristic, 
As a result, our analysis focuses on the dynamic power, 
which is proportional to the product of clock rate $f$, logic switching rate $\alpha$ and the load $C_L$ of each cell.
\\\indent
Figure~\ref{chart:power} shows the power dissipation improvement of DeAr compared with other architectures.
DeAr saves 22.7\%--15.1\% against VLIW.
%and 21.1\%--16.5\% (BLAS) and 17.3\%--13.1\% (GDSPK) of power compared with the 3-way VLIW.
The power improvement achieved by DeAr is attributed to the low RF access rate achieved by the HDFG-based scheduling, which reduces $\alpha$, 
and the banked RF organization, which reduces $C_L$.
\\\indent
The power dissipation of composite-ALU is significantly influenced by its ALU order.
Below 600 MOPS, DeAr achieves 46.9\%--41.2\% power reduction against AMS, 
but only achieves 7.1\%--2.6\% against MSA.  
Compared with AMS, MSA benefits from high OPC, which reduces $f$, and low RF access rate, which reduces $\alpha$, 
so it stays competitive against DeAr in regard to power dissipation.
However, the sharp area growth of composite-ALU also drives severe growth of $C_L$.
As a result, the power dissipation of MSA gradually diverges from the one of DeAr as throughput increasing, 
and DeAr achieves 22.5\%--8.7\% power improvement against MSA while throughput is above or equal to 600 MOPS.
\\\indent
Although the scalar processor cost low area, which implies low $C_L$, 
the worst OPC and RF access rate among designs brings unfavorable $f$ and $\alpha$, 
which result in severe power dissipation. 
Compared with the scalar, DeAr saves 52.1\%--45.8\% of power while below 700 MOPS, 
and the synthesis for scalar fails while above 700 MOPS.
In regard to power dissipation, DeAr achieves the best result among designs regardless throughput.%
\setlength{\textfloatsep}{10pt}% Remove \textfloatsep
\begin{figure}
\begin{center}
\subfigure[b][Area improvement]
{
\label{chart:area}
\includegraphics[width=0.25\textwidth]{charts/area_average_save.eps}%
}%
\subfigure[b][Power dissipation improvement]
{
\label{chart:power}
\includegraphics[width=0.25\textwidth]{charts/power_average_save.eps}
}
\end{center}
\caption{Performance improvement of DeAr}
\label{chart:area}
\end{figure}
